\pdfbookmark{Abstract}{abstract}
\chapter*{Abstract}\label{chapter:abstract}
The \ac{IoT} is one of the biggest topics in the recent years.
In order to limit the vast area of \ac{IoT}, more and more standards are defined and subtopics established.
The Industry 4.0, that includes smart factories, is one of them, with the goal to improve the value chain in a factory, enable process automation and adding intelligent connections between different companies and units.
The fundamental architecture that is necessary in this context is a highly distributed architecture, that can have thousands of nodes with multiple sensors, machines or smart components connected to them.

The concept of Fog Computing bring the cloud and the related functionalities as close to the underlying network as possible, to create smaller independent fog clouds.
Virtual Machines are a common approach in cloud systems, but due to the fact that in an \ac{IoT} context mostly low power devices are used, these technology is not feasible in any case.
In addition, in some scenarios these devices have to interact in a highly heterogeneous and hybrid environment.
Lossy signals and short range radio technologies are widely used and nodes can appear and disappear frequently.
The latency has to keept low to enable real-time applications, the traffic and resource overhead must be as small as possible and for security and data privacy reasons, sensitive data should keept on-premise.

This thesis describes an approach to design and implement a fog service orchestration engine for smart factories.
The aim of this work is to create a prototypical implementation of an orchestration engine for a fog node, called Motey, that can deploy Docker containers to the same node or on any other fog node in the cluster.
Furthermore, the prototype should consider specific functional and non-functional constraints while deploying the containers.
A condition can be a hardware requirement, a required software or a dependency to another node.

One of the biggest challenge in this project was to create a fast and lightweight connection between the nodes, that was achieved by the ZeroMQ protocol.
The node discovery, that enables each node to have knowledge of all the others, is realized with the \acs{MQTT} protocoal and is used to establish the mentioned inter node connection.
The abstraction of the underlying virtualization tools partly kept unsolved, because only a few virtualization tools supports the ARM \acs{CPU} architecture.

Motey can be seen as a good starting point for a complex environment made for fog computing.
It has also space for some improvements, for example the autonomous behavior can be improved a lot.
One possibility could be ability to respond to real-time requirements even in case of the absence of the centralized cloud level.
At the end the created project successfully reveal that the developed concept works out pretty well in a prototypical quality.
Motey is a well developed, tested and documented and can be considered as a pretty solid basis for further development.
