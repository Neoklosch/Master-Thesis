\phantomsection
\addcontentsline{toc}{chapter}{Acronyms}
\renewcommand\refname{Acronyms}
\section*{Acronyms}
\begin{acronym}[NFV-MANO] % längste Abkürzung steht in eckigen Klammern
    \setlength{\itemsep}{-\parsep} % geringerer Zeilenabstand

    % A
    \acro{AE}{Autoscaling Engine}
    \acro{AES}{Advanced Encryption Standard}
    \acro{AMQP}{Advanced Message Queuing Protocol}
    \acro{API}{Application Programming Interface}
    % B
    \acro{BSS}{Business Support Systems}
    % C
    \acro{CapEx}{Capital Expenditure}
    \acro{CI}{Continuous Integration}
    \acro{CLI}{Command Line Interface}
    \acro{CPU}{Central Processing Unit}
    \acro{CPS}{Cyber-Physical System}
      \acroplural{CPS}[CPSs]{Cyber-Physical Systems}
    \acro{CS}{Cyber System}
    % D
    % E
    \acro{EPGM}{Encapsulated Pragmatic General Multicast}
    \acro{EMS}{Element Management System}
    \acro{ETSI}{European Telecommunications Standards Institute}
    % F
    \acro{FM}{Fault Management}
    \acro{FOKUS}{Fraunhofer-Institut für Offene Kommunikationssysteme}
    % G
    \acro{GUI}{Graphical User Interface}
    % H
    \acro{H2H}{Human-to-Human}
    \acro{H2M}{Human-to-Machine}
    \acro{HATEOAS}{Hypermedia As The Engine Of Application State}
    \acro{HTTP}{Hypertext Transfer Protocol}
    % I
    \acro{IERC}{European Research Cluster on the Internet of Things}
    \acro{IoE}{Internet of Energy}
    \acro{IoP}{Internet of People}
    \acro{IoS}{Internet of Services}
    \acro{IoT}{Internet of Things}
    \acro{IIoT}{Industrial Internet of Things}
    \acro{IP}{Internet Protocol}
    \acro{IPC}{Inter-Process Communication}
    \acro{IT}{Information Technology}
    % J
    \acro{JSON}{JavaScript Object Notation}
    \acro{JVM}{Java Virtual Machine}
    % K
    % L
    \acro{LXC}{Linux Containers}
    % M
    \acro{M2M}{Machine-to-Machine}
    \acro{MANO}{Management And Orchestration}
    \acro{MQTT}{Message Queue Telemetry Transport}
    % N
    \acro{NF}{Network Function}
      \acroplural{NF}[NFs]{Network Functions}
    \acro{NFV}{Network Function Virtualistion}
    \acro{NFVI}{Network Function Virtualization Infrastructure}
    \acro{NFV-MANO}{Network Function Virtualistion Management And Orchestration}
    \acro{NFVO}{Network Function Virtualistion Orchestrator}
    \acro{NIST}{National Institute of Standards and Technology}
    % O
    \acro{OASIS}{Organization for the Advancement of Structured Information Standards}
    \acro{OpEx}{Operating Expense}
    \acro{OS}{Operating System}
    \acro{OSS}{Operations Support Systems}
    % P
    \acro{PGM}{Pragmatic General Multicast}
    \acro{PNF}{Physical Network Function}
    \acro{PoP}{Point of Presence}
      \acroplural{PoP}[PoPs]{Points of Presence}
    % Q
    \acro{QoS}{Quality of Service}
    % R
    \acro{RAM}{Random Access Memory}
    \acro{REST}{Representational State Transfer}
    \acro{RFID}{Radio Frequency Identification}
    % S
    \acro{SDN}{Software Defined Networking}
    \acro{SSL}{Secure Sockets Layer}
    % T
    \acro{TCP}{Transmission Control Protocol}
    \acro{TLS}{Transport Layer Security}
    \acro{TOSCA}{Topology and Orchestration Specification for Cloud Applications}
    % U
    \acro{UI}{User Interface}
    \acro{URL}{Uniform Resource Locator}
    % V
    \acro{VAL}{Virtualization Abstraction Layer}
    \acro{VIM}{Virtual Infrastructure Manager}
      \acroplural{VIM}[VIMs]{Virtual Infrastructure Managers}
    \acro{VM}{Virtual Machine}
      \acroplural{VM}[VMs]{Virtual Machines}
    \acro{VMM}{Virtual Machine Monitor}
    \acro{VNF}{Virtual Network Function}
      \acroplural{VNF}[VNFs]{Virtual Network Functions}
    \acro{VNFM}{Virtual Network Function Manager}
    % W
    % X
    % Y
    \acro{YAML}{YAML Ain\'t Markup Language}
    % Z
\end{acronym}
