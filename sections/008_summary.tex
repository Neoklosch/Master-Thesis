\chapter{Conclusion}
\label{chapter:conclusion}
\minitoc\vspace{.5cm}
This final chapter sums up the thesis as well as the created prototype called Motey.
The first section outlines the initial idea behind the project and reprocues the different work stages.
In section \ref{section:summary_impact} the usage of the project is shown and the uses cases is elaborated.
Issues that remained unsolved are described in the following section.
At the end the impact for the industry and an outlook for the Motey project is shown.

\section{Summary}
Objective of this thesis was the "Design and Development of a Fog Service Orchestration Engine for Smart Factories".
Therefore a prototype for such an orchestration engine called Motey was created.
The application was created in cooperation with the \acf{FOKUS}.
The main idea of the project was worked out in an iterative process together with supervisors of the \ac{FOKUS}.
In several meetings the objectives was analyzed and elaborated.
Motey is designed to be executed on each fog node and is able to instanciate an inter-node connection.
Each node has knowledge of all the other nodes and can communicate with them right at the moment they have to.\newline

The application is also able to handle different so called capabilites.
That are functional and non-functional requirements a node can fulfill.
These capabilities are used to deploy images, for example \acp{NF} to a one or more nodes.
The node that receives a deployment schema check the requirements of the contained images and deploy them on the same node if it is possible or on other nodes that are able to deploy them.
Labels can be configured from within the node itself or from any external application.
Finally an image will be started via the related virtualization tool.
In case of the prototype and as a primary requirement of this project, Docker is the default engine.
For the inter-node communication the ZeroMQ and \ac{MQTT} protocols are used.
They are much faster than the pretty famous \ac{HTTP} protocol, as shown in the performance evaluation section \ref{section:performance-evaluation}.
An \ac{HTTP} \ac{REST} \ac{API} is also available to enable compatibility with well known cloud orchestration engines like Open Baton or OpenStack.

The whole application is well documented and tested.
Each component has a related unit test and the documentation is available online at \url{https://neoklosch.github.io/Motey/}.
A \ac{CI} pipeline is used to automatically build each new version, test them and finally deploy them as a Docker image to the Docker Hub.
This is espacially useful to reduce the overhead of repeating tasks, to guarantee the correctness of the current version and also to have an up-to-date version available via the Docker infrastructure.

\section{Dissemination}
% integration in open baton denkbar
% als weiterentwicklung für arbeiten im bereich autonomous deployment in fog
% Smart Factories
% Papers planed (?)

% Who uses your component or who will use it? Industry projects, EU projects, open
% source...? Is it integrated into a larger environment? Did you publish any papers?
\doit

\section{Impact}
\label{section:summary_impact}
% Summarize the main problems. How did you solve them? Why didn’t you solve them?

% node discovery
% -> central broker where each node have to registered to
% -> information on each new node

% virt on low power devices
% -> docker
% -> XEN & Co not possible

% capabilities
% -> external API
% ---> ZeroMQ

% Inter node connection
% -> node discovery + ZeroMQ
\doit

\section{Outlook}
% Future work will enhance Component X with new services and features that can be used

% basic well developed fundament
% a lot of space to improve, because complex topic
% more autonomous
% -> if centralized level is absent, still do some shit
% access control / right management / master node
% node discovery
% -> master node can be the new MQTT broker
% capability management
% -> final integration with hardware component
% -> autonomous behavior if one label disappers -> container will be deployed on different node
% full integration and test with open baton
\doit
