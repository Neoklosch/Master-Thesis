\chapter{Implementation}\label{chapter:implementation}
% This chapter describes the implementation of component X. Three systems were chosen
% as reference implementations: a desktop version for Windows and Linux PCs, a Windows
% Mobile version for Pocket PCs and a mobile version based on Android.
This chapter describes the implementation of docker plugin as well as the constraint logic.
Used technologies, libraries and tools, as well as custom components would be presented and the functionality will be demonstrated.
Thereby challenges and problems during the development of the plugins will be shown and the solutions will be discussed.

\section{Environment}
% The following software, respectively operating systems, were used for the implementation:
% • Ubuntu
% • Python 3.5
% • Docker
% • XY Library
% DI, RxPy,
\doit

\section{Project structure}
% The implementation is seperated into 2 distinguished eclipse projects as depicted in
% figure 5.1.
% The following listing briefly describes the single packages of both projects in alphabetical
% order to give an overview of the implementation:

% Verzeichnisse erklären
\begin{itemize}
  \item{\textbf{Directory: ./}} has all the necessary configuration files for the different services.
  \begin{itemize}
    \item{\textbf{File: .dockerignore}} is used during the build process of a Docker image. Excludes several folders during the build phase.
    \item{\textbf{File: .editorconfig}} contains informations for \acp{IDE} and editors to guarantee a consistent coding style.
    \item{\textbf{File: .gitignore}} excludes file to be tracked by the version control system git.
    \item{\textbf{File: .travis.yml}} is used by the continuous integration tool Travis CI.
    \item{\textbf{File: AUTHORS.rst}} a list of all contributors.
    \item{\textbf{File: CHANGELOG.rst}} this document records all notable changes to the motey engine.
    \item{\textbf{File: LICENSE}} the License of the project (Apache License Version 2.0).
    \item{\textbf{File: main.py}} can be used to start motey in debug mode.
    \item{\textbf{File: MANIFEST.in}} contains meta information for the python setup procedure.
    \item{\textbf{File: README.rst}} file to show up a short documentation on github and will act as the starting point of the project.
    \item{\textbf{File: setup.py}} will be used to install motey on a local machine.
  \end{itemize}
  \item{\textbf{docs:}} contains the files to create and display the documentation resources.
  \begin{itemize}
    \item{\textbf{Directory: source}} has all the files to autogenerate the documentation files from the source code.
    \item{\textbf{File: Makefile}} this file was created by the Sphinx documentation tool. By executing the \textit{Makefile} the related documentation files will be created, the current branch will be switched to \textit{gh-pages} which is be used to display the github page at \url{https://neoklosch.github.io/Motey/} and a new commit will be pushed. Finally the branch will be switched back again.
  \end{itemize}
  \item{\textbf{motey:}} this folder contains the motey main engine. It is the main python project. The whole structure will be explained on the next pages in detail.
  \item{\textbf{motey-docker-image:}} has all the necessary files to create a Docker image.
  \begin{itemize}
    \item{\textbf{File: Dockerfile}} to build the Docker image. Is analogous to a Makefile but can only be used by the Docker engine.
    \item{\textbf{File: setup.sh}} will be executed during the build phase and will install necessary tools and can executed command line instructions.
    \item{\textbf{File: requirements.txt}} a list with the python requirements which are necessary to run the motey engine and which should be installed during the build phase via pip.
  \end{itemize}
  \item{\textbf{resources:}} is a resource folder for the github documentation. Will only be used by the \textit{README.rst} file in the root folder and the \textit{index.rst} file in the docs\/source folder.
  \item{\textbf{samples:}} contains some samples to test the functionality of the motey engine. Is primarily a playground to test new functions.
  \item{\textbf{scripts:}} some scripts which will be executed frequently during the development phase.
  \begin{itemize}
    \item{\textbf{Folder: config}} configuration files which could be used for the mosquitto \ac{MQTT} broker Docker image.
    \item{\textbf{File: addcapability.py}} can be used to add new capability entries to a running motey instance.
    \item{\textbf{File: start\_test\_setup.sh}} can be used to start a new local Docker test cluster.
  \end{itemize}
  \item{\textbf{tests:}} contains all the unit test which are executed by the continuous integration script and the python setup procedure.
  \item{\textbf{webclient:}} this folder contains the \ac{GUI} for the motey engine. Will also be described on the next pages in detail.
\end{itemize}

\subsection{motey engine}
% klassendiagramm
% einzelne Pakete erklären

\section{Used external libraries}
\paragraph{daemonize}
\paragraph{dependency-injector}
\paragraph{docker sdk}
\paragraph{Flask}
\paragraph{Logbook}
\paragraph{paho-mqtt}
\paragraph{pyzmq}
\paragraph{tinydb}
\paragraph{Yapsy}
\doit

\section{Important Implementation Aspects}
% Do not explain every class in detail. Give a short introduction about the modules or
% the eclipse projects. If you want to explain relevant code snippets use the ’lstlisting’ tag
% of LaTeX. Put only short snippets into your thesis. Long listing should be part of the
% annex.

% You can also compare different approaches. Example: Since the implementation based
% on X failed I choosed to implement the same aspect based on Y. The new approach
% resulted in a much faster ...

% Core samt daemon erklären
\doit

\section{Implementation of the data layer}
% eine Database für labels, services und nodes
% json models aufzeigen
\doit

\section{Implementation of the orchestration layer}
\doit

\section{Implementation of the virtualization layer}
\doit

\section{Implementation of the communication layer}
\doit

\section{Implementation of the capability management}
\doit

\section{Implementation of the user interface}
\doit

\section{Deployment and Continous Integration}
% Travis.io
% Docker Container + Docker Hub
% setup.py
% Sphinx
\doit

\section{Conclusion}
\doit
