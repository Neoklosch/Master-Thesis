\acresetall

\chapter{Implementation}\label{chapter:implementation}
% This chapter describes the implementation of component X. Three systems were chosen
% as reference implementations: a desktop version for Windows and Linux PCs, a Windows
% Mobile version for Pocket PCs and a mobile version based on Android.
This chapter describes the implementation of docker plugin as well as the constraint logic.
Used technologies, libraries and tools, as well as custom components would be presented and the functionality will be demonstrated.
Thereby challenges and problems during the development of the plugins will be shown and the solutions will be discussed.

\section{Environment}
% The following software, respectively operating systems, were used for the implementation:
% • Windows XP and Ubuntu 6
% • Java Development Kit (JDK) 6 Update 10
% • Eclipse Ganymede 3.4
% • Standard Widget Toolkit 3.4
\doit

\section{Project structure}
% The implementation is seperated into 2 distinguished eclipse projects as depicted in
% figure 5.1.
% The following listing briefly describes the single packages of both projects in alphabetical
% order to give an overview of the implementation:
\doit

\section{Used external libraries}
\doit

\section{Important Implementation Aspects}
% Do not explain every class in detail. Give a short introduction about the modules or
% the eclipse projects. If you want to explain relevant code snippets use the ’lstlisting’ tag
% of LaTeX. Put only short snippets into your thesis. Long listing should be part of the
% annex.

% You can also compare different approaches. Example: Since the implementation based
% on X failed I choosed to implement the same aspect based on Y. The new approach
% resulted in a much faster ...
\doit

\section{Implementation of the orchestration layer}
\doit

\section{Implementation of the constraint layer}
\doit

\section{Implementation of the user interface}
\doit

\section{Conclusion}
\doit
