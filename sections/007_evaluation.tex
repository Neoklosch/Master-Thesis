\chapter{Evaluation}
\label{chapter:evaluation}
\minitoc\vspace{.5cm}
% In this chapter the implementation of Component X is evaluated. An example instance
% was created for every service. The following chapter validates the component implemented
% in the previous chapter against the requirements.
% Put some screenshots in this section! Map the requirements with your proposed solution.
% Compare it with related work. Why is your solution better than a concurrent
% approach from another organization?
In this chapter the implementation of the plugins, based on the previously defined requirements and concepts, are evaluated.
Afterwards a demonstration of the system will be shown.

\section{Test Environment}
\label{section:test-environment}
Motey and the performance scripts in section \ref{section:performance-evaluation} was tested on three different devices.
\begin{enumerate}
  \item \textbf{Raspberry Pi 2 Model B Version 1.1} hereinafter called \textbf{NeoPi}, that has a ARM Cortex-A7 \ac{CPU} with four cores each with 900 MHz and a 32-bit architecture. It has 1024 MB \ac{RAM} and a 10/100-MBit-Ethernet port. The device is running a Raspbian 8 (jessie) with Linux Kernel version 4.9.
  \item \textbf{Raspberry Pi 3 Model B Version 1.2} hereinafter called \textbf{BuffPi}, is the currently the newest Raspberry Pi on the market. It has a ARM Cortex-A53 \ac{CPU} also with four cores, but each has 1200 MHz and 64-bit architecture. It also has 1024 MB \ac{RAM} and a 10/100-MBit-Ethernet port.  This devices is also runngin Raspbian 8 (jessie) with Linux Kernel version 4.9.
  \item \textbf{Acer Aspire V5-573G} hereinafter called \textbf{Laptop}, with an Intel Core i7-4500U \ac{CPU} with two cores each 1.80 GHz and a 64-bit architecture as well. It has 8 GB \ac{RAM} and a 802.11n WiFi connection. It is running a Linux Mint 18.1  64-bit \ac{OS} with a Linux Kernel version 4.4.
\end{enumerate}

Both the \textit{NeoPi} and the \textit{BuffPi} are connected via ethernet to a router.
The \textit{Laptop} is connect via 802.11n WiFi to the router.
% todo: network information, multiple devices, etc.

\section{Performance Evaluation}
\label{section:performance-evaluation}
% Docker vs. VMWare
% there are several performance comparisons out there: for example \autocite{IBM:Performance:2014} or \autocite{Hussain:Performance:2014} or \autocite{Lloyd:Performance:2015}
% hard to compare on a raspberry pi because of the ARM architecture
% -> not that much virtualization tools supports ARM
% Encryption vs. non Encryption
% Performance ZeroMQ
% ZeroMQ vs HTTP
% Performance MQTT
% MQTT vs HTTP
\doit

\section{Scalability}
\label{section:scalability}
% MQTT maximum
% weitere nodes hinzufügen
% existierende entfernen
% Docker Container maximum
\doit

\section{Code Verification}
\label{section:code-verification}
Code verification is a quite important technique in every development project.
There are several possibilities to check and verify the integrity of the created source code.
Two of them are used in this project.
The style check is used to ensure the compliance with the code style guidelines and the unit tests are used to verify the correctness of the code and the outcome of the functions itself.
As the guidelines the pretty famous \ac{PEP} 8 style guide\footnote{\url{https://www.python.org/dev/peps/pep-0008}} is used.
\ac{PEP} 8 is used in the Python standard library code and is well established in the community.
A standardized code style is recommended in team projects as well as in open source projects.
But also in private project a commitment to a specific style can be helpful to make the project easier to maintain.
The code in general becomes more readable, more understandable and the amount of errors can be decreased.
Due to the fact that overall code is read more often than it is written, other people will be satisfied by having a common understanding of the "grammar" of the code to be used.
As the tool of choice the library \textit{pycodestyle} will be used.
The style check is part of the \ac{CI} pipeline and the concrete implementation was shown in listing \ref{code:travis_config}.
If there is any style check violation, the \ac{CI} build process will fail and the new Docker image will not be uploaded.

\begin{listing}[H]
  \begin{minted}{bash}
  $ pycodestyle --ignore=E241,E501 motey/
  motey/di/app_module.py:48:38: E128 continuation line under-indented for visual indent
  motey/di/app_module.py:49:38: E128 continuation line under-indented for visual indent
  motey/di/app_module.py:50:38: E128 continuation line under-indented for visual indent
  motey/di/app_module.py:51:38: E128 continuation line under-indented for visual indent
  motey/di/app_module.py:52:38: E128 continuation line under-indented for visual indent
  motey/di/app_module.py:53:38: E128 continuation line under-indented for visual indent
  motey/di/app_module.py:54:38: E128 continuation line under-indented for visual indent
  motey/configuration/configreader.py:6:66: E703 statement ends with a semicolon

  The command "pycodestyle --ignore=E241,E501 motey/" exited with 1.

  $ pycodestyle --ignore=E241,E501 samples/

  The command "pycodestyle --ignore=E241,E501 samples/" exited with 0.
  \end{minted}
  \caption{Sample output of the style check validation from the Travis \ac{CI} build process number 148\autocite{Travis:Build:148}}
  \label{code:style_check_validation}
\end{listing}

Listing \ref{code:style_check_validation} shows an example output of the Travis \ac{CI} build process.
The first style check call fails due to multiple validations, but the second one exited successfully.\newline

The second verification is are the unit tests.
A unit test is as the name implies a software test method which tests each component of an application.
Each component should be tested as isolated as possible by invoking only one or a couple of methods from a unit and the result should be verified automatically and compared with an expected result.\autocite[cf.][p. 320]{Olan:2003:UTT:948785.948830}
All the used objects in a class should be as independent from each other as possible.
To ensure this, the objects have to be mocked by a mocking framework.
The python unit test framework has a build-in mocking framework since version 3.
A mock is a fake objects that acts as a dummy for the class to be tested.
To decouple the dependencies and to increase the testability and more specific to make used object easier mockable, the \ac{DI} pattern is used.
Each injected object can be easily mocked outside of the class.
Due to the fact that Python allows to modify each class member at any time, \ac{DI} is not really necessary, but it helps to make it clearer to understand and the code more readable and maintainable.\newline

The advantages of unit tests in general are that problems can be easier localized and much faster detected.
If an error occures during unit testing a former well working code, it is obvious that the last code changes are the reason for the error.
This benefit is also helpful to readuce the fear of refactoring or extend code parts.
Even pretty old parts of an application can be modified without the risk of any major problems.
When the unit test are part of the \ac{CI}, each build will be checked.
This reduces the risk of the deployment of faulty code.
And finally unit tests can be a good starting point for new team members, because it helps to understand the functionality of a class.
An unit test can acts as some kind of documentation.
The downsides are that unit tests are coded.
This means also unit tests can have errors.
Furthermore poorly written unit tests or tests that a written unpleasured can end up in a wrong feeling of safeness.
Some error case could not be catched and the submitted code is still faulty.
Additionally the test setup is not realistic.
Integration test are more accurate in terms of the human behavior.
Also the dependencies between the component and the interaction between them can be tested much better with integration test.
The most critical part of unit test in the economy is that the take time.
For each written component, the realted tests take a significant time to be developed.
In a real life project this could be unacceptable even if it is important to have them.\newline

\begin{listing}[H]
  \begin{minted}{python}
  class TestServiceRepository(unittest.TestCase):
    @classmethod
    def setUp(self):
        self.text_service_id = uuid.uuid4().hex
        self.test_service = {'id': self.text_service_id, 'service_name': 'test service name', 'images': ['test image']}
        service_repository.config = {'DATABASE': {'path': '/tmp/testpath'}}
        service_repository.BaseRepository = mock.Mock(service_repository.BaseRepository)
        service_repository.TinyDB = mock.Mock(TinyDB)
        service_repository.Query = mock.Mock(Query)
        self.test_service_repository = service_repository.ServiceRepository()

    def test_has_entry(self):
        self.test_service_repository.db.search = mock.MagicMock(return_value=[1, 2])

        result = self.test_service_repository.has(service_id=self.test_service['id'])

        self.assertTrue(self.test_service_repository.db.search.called)
        self.assertTrue(result)
  \end{minted}
  \caption{Extract from the Motey unit test of the ServiceRepository}
  \label{code:sample_unit_test}
\end{listing}

% TODO
\todo{describe the listing}\newline

Unit testing in general can be realized in two different ways.
The first possibility is the by writing the tests after the code is done.
\todo{finalize this paragraph}\newline

The second possibility is the \ac{TDD}, that is related to the test-first programming concept of the extreme programming development.
In this development methodology the tests are written before the implementation of a class is created.
This helps to plan the architecture of a class and catch edge cases before the implementation is done.
\ac{TDD} is an iterative process where at first the test is written, then the test will be executed and must fail, afterwards the code is written and the tests should be executed again, but this time successfully.
Finally the process can be repeated for example if the code has to be refactored or extended.
Especially in an extrem programming environment \ac{TDD} in combination with pair programming and code reviews are reasonable.\newline

Motey was created with the "normal" unit testing approach, due to the fact that it was a one man project with rapidly changing requirements and an explorative approach to create the prototype.
Nevertheless both methods ends up in a well tested code base and an assured code stability.

\section{Conclusion}
\doit
