\acresetall

\chapter{Requirements Analysis}\label{chapter:requirements-analysis}
% This section determines the requirements necessary for X. This includes the functional
% aspects, namely Y and Z, and the non functional aspects such as A and B.
Based on the fundamentals the requirements for the plugin to be developed will be formulated in this chapter.
Thereby aspects that will be relevant for the specific implementation will be considered.

% In this chapter you will describe the requirements for your component. Try to group the
% requirements into subsections such as ’technical requirements’, ’functional requirements’,
% ’social requirements’ or something like this. If your component consist of different partial
% components you can also group the requirements for the corresponding parts.
% Explain the source of the requirements.
% Example: The requirements for an X have been widely investigated by Organization
% Y.
% In his paper about Z, Mister X outlines the following requirements for a Component
% X

\section{Functional requirements}
As the fundamental condition, the prototype to be developed has to be based on the Open Baton Framework.
Open Baton inherently supports OpenStack as the \ac{ETSI} \ac{MANO} \ac{VIM} layer.
By default OpenStack useses virtual machines to enable facilitate the virtualization of the \acp{NF}.
This is a rock solid solution for a cloud environment.
Unfortunately a bare-metal virtualization is most of the times not feasible on small power devices like they are used in the \ac{IoT} area.
Therefore OpenStack should be replaced by a more efficient and lightweight solution like container virtualization.
% why?

% images should be found by the registry?
% constraints noch unklar
% based on schema on clients/NVFs/Server? -> have to be installed before everything else?
% based on realtime detection on nodes?
% should only the server have these information or should the node himself have the infromations/schema and can provide it?

\section{Technical requirements}
After the general conditions for the plugin has been defined, the technological requirements will also be addressed.
When selecting the programming environment special attention is paid to the compatibility of the Open Baton framework.
This is particularly important as the effort to develop a prototype should be a s small as possible and the plugin should be immediately applicable.
Open Baton has an build in plugin mechanism, which allows to replace for example the \ac{VIM} driver or the \ac{VNFM}.
Because Open Baton is open source, it would be also feasible to extend the system if necessary, but with the addition, that the system still follows the \ac{ETSI} \ac{MANO} conventions as a secondary goal.

Open Baton also provides a \ac{GUI} as well as a command line tool to control the system.
Even the plugin and all the related functionalities should be controlled and monitored via the given tools.
This shortening the learning curve and reducing training demands for the enduser.
The system can be used in the same way as the user is used to.

% Open Baton has to be configured
% the plugins have to be installed (VIM Driver, VNFM(?))
% Docker has to be installed
% node must be able to execute docker and open baton client software

% virtualization
% constraint

\section{Use-Case-Analysis}
\doit
% Deploy VNFs in factory (e.g. mass production, tracking of stuff)
% ad hoc deployment if truck arives at factory
% detect the right nodes, e.g. software needs specific sensor or latency

\section{Delineation from existing solutions}
\doit

% Kubernetes and Docker Swarm
% Cloudify (what else?)
% Open Baton with OpenStack

\section{Conclusion}
\doit
