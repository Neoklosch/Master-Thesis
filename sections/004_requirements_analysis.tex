\chapter{Requirements Analysis}
% This section determines the requirements necessary for X. This includes the functional
% aspects, namely Y and Z, and the non functional aspects such as A and B.
\doit

\section{Overview}
% In this chapter you will describe the requirements for your component. Try to group the
% requirements into subsections such as ’technical requirements’, ’functional requirements’,
% ’social requirements’ or something like this. If your component consist of different partial
% components you can also group the requirements for the corresponding parts.
% Explain the source of the requirements.
% Example: The requirements for an X have been widely investigated by Organization
% Y.
% In his paper about Z, Mister X outlines the following requirements for a Component
% X
\doit

\section{Technical requirements}
% The following subsection outlines the technical requirements to Component X.
\doit

\section{Technologies}
\doit

\section{Use-Case-Analysis}
\doit

\section{Delineation from existing solutions}
\doit

\section{Conclusion}
\doit
