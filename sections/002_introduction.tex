\chapter{Introduction}

\section{Background and Motivation}

% Research Context - Forschungskontext
% -- IoT
The \ac{IoT} is one of the biggest topic in the recent years.
Companies with a focus in that area have an enormous market growth with plenty of new opportunities, use cases, technolgies, services and devices.
Bain \& Company predicts an annual revenue of \$450 billion for companies who selling hardware, software and comprehensive solutions in the \ac{IoT} context by 2020.\cite{Bosche:2016}
In order to limit the vast area of \ac{IoT}, more and more standards are defined and subtopics established.
The \ac{IERC} devided them into eight categories: Smart Cities, Smart Healthcare, Smart Transport and Smart Industry also known as Industry 4.0 to mention only a few.
All of them are well connected, for example a Smart Factory, which is a part of the Smart Industry, can get a delivery from a self driving truck (Smart Transport) which navigates through a Smart City to get to the factory.
Such information networks are one of the main goals of \ac{IoT}.
% Research Area - Forschungsgebiet
% -- Smart Factories
In the Industry 4.0 for example multiple Smart Factories should be interconnect into a distributed and autonomous value chain.
Also the automation in a single factory will be increased which helps to have a more flexible and efficient production process.
Currently a factory has a high degree of automation, but due to a lack of intelligence and communication between the machines and the underlying system, they can not react to changing requirements or unexpected situations.
% Application Area - Anwendungsbereich
% -- CS & CPS
One solution to achieve that are \acp{CPS}.
These are virtual systems which are connected with embedded systems to monitor and control physical processes.\cite{Lee:2008}
A normal \acp{CS} is passive, means it could not interact with the physical world, with the appearance of \acp{CPS} things can communicate so the system has significantly more intelligence in sensors and actuators.\cite{Poovendran:2010}

% Research Focus - Forschungsschwerpunkt aka Praxisproblem
% -- Virtualization, Orchestration, Deployment of functions
\todo{Cloud Computing was one of the prime topic in the recent years.
It changed from a monolithic to more distributed multicloud architecture.
With the appearance of the Internet of Things (IoT) and related architectures like Fog Computing the cloud moves away from centralized data centers to the edge of the underlying network [1].
Such a network can have thousands of nodes with multiple sensors, machines or smart components connected to them.
An ”intermediate layer be- tween the IoT environment and the Cloud” [2] enables a lot of new possibilities like pre-computation and storage of gathered data, which reduces traffic and resource overhead in the cloud, it keeps sensitive data on-premise [2] and enables real-time applications to take decisions based on analytics running near the device and a lower latency [3].
On the other hand there are also a lot of challenges in these highly heterogeneous and hybrid environment.
As an example in some scenarios multiple low power devices have to interact with each other, lossy signals and short range radio technologies are widely used and nodes can appear and disappear frequently [3].
Especially the last case is elaborated because the underlying system has to handle that.
Furthermore the required applica- tions running on these nodes can be change commonly and have to be deployed and removed in a dynamical way.
Virtualization with Virtual Machines (VMs) is a common approach in Cloud systems to provide elasticity of large-scale shared resources[4].
A more lightweight, less resource and time con- suming solution is container virtualization.
”Furthermore, they are flexible tools for packaging, delivering and orchestration software infrastructure services as well as application”[4].
Orchestration tools like Kubernetes[5], Docker Swarm[6] or CoreOS[7] which deploy, scale and manage containers to clusters of hosts have become established in the last years.
Moving this technology over to the IoT area many challenges can be solved.
Dynamically deployed applications at the edge of a network can store and preprocesses gather data even if a node have no connection to the cloud because of lossy signals.
Traffic can be reduced by only transmitting aggregated data back to the cloud.
More often small low-power devices with limited computational power are be used as IoT nodes which also profit rather from lightweight container solutions than from resource consuming VMs.
This paper shows the capabilities of container orchestration for the IoT and Smart Factories using the Open Baton frame- work.
Therefor a plugin will be created which can orchestrate Docker containers based on functional and non-functional constraints to fog nodes.}


% Taxonomy - Taxonomie

\section{Problem Statement}

% State of the Art
% -- was gibt es bisher
% --- Container in der Cloud
% --- NFO

% write about the first issue
% -- deployment of applications

% write about the second issue
% -- detect constraints

% write about the synopsis (Zusammenfassung) of the issues

\section{Assumptions and Scope}

% write about the research assumptions (Forschungsannahmen)
% -- haypothese, was ist der erwartete outcome?

% write about the research scope (Forschungsbereich) — Figures 1.1, 1.2 and 1.4
% -- was ist drin, was nicht

\section{Objectives and Contributions}

% Research Objectives & Contributions
% Ziele und Beitrag für die Wissenschaft

\section{Methodology and Outline}

% write about the research outline and Figure 1.6. Summarize Chapters 1 to 8
