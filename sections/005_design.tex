\acresetall

\chapter{Concept}\label{chapter:concept}
% This chapter introduces the architectural design of Component X. The component consists
% of subcomponent A, B and C.
% In the end of this chapter you should write a specification for your solution, including
% interfaces, protocols and parameters.
This chapter introduces the architectural design of plugin respect to the previously defined requirements.
Therefore the used development environment will be analyzed.
Followed by the initial architecture of the plugin to be developed.
Based on that the different layer of the system will be elaborated.

\section{Overview}
% The concept chapter provides a high-level explanation of your solution. Try to explain
% the overall structure with a picture. You can also use UML sequence diagrams for
% explanation.
% Figure 4.1 illustrates the situation between Alice and Bob. (sequence diagram from
% www.websequencediagrams.com)
\doit

% Docker Plugin as Vim Driver

\section{Development environment}
\doit

Open Baton is written in Java and is also partly ported to python and go.

\textbf{Java:}

\textbf{Python:}

\textbf{Go:}

Beside the fact that there is no complete implementation in the latter, the plugin will be written in Java.

The \ac{GUI} is a fundamental web client, which uses Angular.js as its main framework and some smaller tools like jQuery or the pretty famous bootstrap framework.

% Docker as container virtualization tool -> important: analysis for performance Docker vs. Hypervisor Virt.
% yaml for schema

\section{Architecture of the system}
\doit


\subsection{Orchestration layer}
\doit

\subsection{Constraint layer}
\doit

\subsection{User interface}
\doit

\section{Conclusion}
\doit
